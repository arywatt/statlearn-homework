\documentclass[]{article}
\usepackage{lmodern}
\usepackage{amssymb,amsmath}
\usepackage{ifxetex,ifluatex}
\usepackage{fixltx2e} % provides \textsubscript
\ifnum 0\ifxetex 1\fi\ifluatex 1\fi=0 % if pdftex
  \usepackage[T1]{fontenc}
  \usepackage[utf8]{inputenc}
\else % if luatex or xelatex
  \ifxetex
    \usepackage{mathspec}
  \else
    \usepackage{fontspec}
  \fi
  \defaultfontfeatures{Ligatures=TeX,Scale=MatchLowercase}
\fi
% use upquote if available, for straight quotes in verbatim environments
\IfFileExists{upquote.sty}{\usepackage{upquote}}{}
% use microtype if available
\IfFileExists{microtype.sty}{%
\usepackage{microtype}
\UseMicrotypeSet[protrusion]{basicmath} % disable protrusion for tt fonts
}{}
\usepackage[margin=1in]{geometry}
\usepackage{hyperref}
\hypersetup{unicode=true,
            pdftitle={Statlearn - homework II},
            pdfborder={0 0 0},
            breaklinks=true}
\urlstyle{same}  % don't use monospace font for urls
\usepackage{color}
\usepackage{fancyvrb}
\newcommand{\VerbBar}{|}
\newcommand{\VERB}{\Verb[commandchars=\\\{\}]}
\DefineVerbatimEnvironment{Highlighting}{Verbatim}{commandchars=\\\{\}}
% Add ',fontsize=\small' for more characters per line
\usepackage{framed}
\definecolor{shadecolor}{RGB}{248,248,248}
\newenvironment{Shaded}{\begin{snugshade}}{\end{snugshade}}
\newcommand{\KeywordTok}[1]{\textcolor[rgb]{0.13,0.29,0.53}{\textbf{#1}}}
\newcommand{\DataTypeTok}[1]{\textcolor[rgb]{0.13,0.29,0.53}{#1}}
\newcommand{\DecValTok}[1]{\textcolor[rgb]{0.00,0.00,0.81}{#1}}
\newcommand{\BaseNTok}[1]{\textcolor[rgb]{0.00,0.00,0.81}{#1}}
\newcommand{\FloatTok}[1]{\textcolor[rgb]{0.00,0.00,0.81}{#1}}
\newcommand{\ConstantTok}[1]{\textcolor[rgb]{0.00,0.00,0.00}{#1}}
\newcommand{\CharTok}[1]{\textcolor[rgb]{0.31,0.60,0.02}{#1}}
\newcommand{\SpecialCharTok}[1]{\textcolor[rgb]{0.00,0.00,0.00}{#1}}
\newcommand{\StringTok}[1]{\textcolor[rgb]{0.31,0.60,0.02}{#1}}
\newcommand{\VerbatimStringTok}[1]{\textcolor[rgb]{0.31,0.60,0.02}{#1}}
\newcommand{\SpecialStringTok}[1]{\textcolor[rgb]{0.31,0.60,0.02}{#1}}
\newcommand{\ImportTok}[1]{#1}
\newcommand{\CommentTok}[1]{\textcolor[rgb]{0.56,0.35,0.01}{\textit{#1}}}
\newcommand{\DocumentationTok}[1]{\textcolor[rgb]{0.56,0.35,0.01}{\textbf{\textit{#1}}}}
\newcommand{\AnnotationTok}[1]{\textcolor[rgb]{0.56,0.35,0.01}{\textbf{\textit{#1}}}}
\newcommand{\CommentVarTok}[1]{\textcolor[rgb]{0.56,0.35,0.01}{\textbf{\textit{#1}}}}
\newcommand{\OtherTok}[1]{\textcolor[rgb]{0.56,0.35,0.01}{#1}}
\newcommand{\FunctionTok}[1]{\textcolor[rgb]{0.00,0.00,0.00}{#1}}
\newcommand{\VariableTok}[1]{\textcolor[rgb]{0.00,0.00,0.00}{#1}}
\newcommand{\ControlFlowTok}[1]{\textcolor[rgb]{0.13,0.29,0.53}{\textbf{#1}}}
\newcommand{\OperatorTok}[1]{\textcolor[rgb]{0.81,0.36,0.00}{\textbf{#1}}}
\newcommand{\BuiltInTok}[1]{#1}
\newcommand{\ExtensionTok}[1]{#1}
\newcommand{\PreprocessorTok}[1]{\textcolor[rgb]{0.56,0.35,0.01}{\textit{#1}}}
\newcommand{\AttributeTok}[1]{\textcolor[rgb]{0.77,0.63,0.00}{#1}}
\newcommand{\RegionMarkerTok}[1]{#1}
\newcommand{\InformationTok}[1]{\textcolor[rgb]{0.56,0.35,0.01}{\textbf{\textit{#1}}}}
\newcommand{\WarningTok}[1]{\textcolor[rgb]{0.56,0.35,0.01}{\textbf{\textit{#1}}}}
\newcommand{\AlertTok}[1]{\textcolor[rgb]{0.94,0.16,0.16}{#1}}
\newcommand{\ErrorTok}[1]{\textcolor[rgb]{0.64,0.00,0.00}{\textbf{#1}}}
\newcommand{\NormalTok}[1]{#1}
\usepackage{graphicx,grffile}
\makeatletter
\def\maxwidth{\ifdim\Gin@nat@width>\linewidth\linewidth\else\Gin@nat@width\fi}
\def\maxheight{\ifdim\Gin@nat@height>\textheight\textheight\else\Gin@nat@height\fi}
\makeatother
% Scale images if necessary, so that they will not overflow the page
% margins by default, and it is still possible to overwrite the defaults
% using explicit options in \includegraphics[width, height, ...]{}
\setkeys{Gin}{width=\maxwidth,height=\maxheight,keepaspectratio}
\IfFileExists{parskip.sty}{%
\usepackage{parskip}
}{% else
\setlength{\parindent}{0pt}
\setlength{\parskip}{6pt plus 2pt minus 1pt}
}
\setlength{\emergencystretch}{3em}  % prevent overfull lines
\providecommand{\tightlist}{%
  \setlength{\itemsep}{0pt}\setlength{\parskip}{0pt}}
\setcounter{secnumdepth}{0}
% Redefines (sub)paragraphs to behave more like sections
\ifx\paragraph\undefined\else
\let\oldparagraph\paragraph
\renewcommand{\paragraph}[1]{\oldparagraph{#1}\mbox{}}
\fi
\ifx\subparagraph\undefined\else
\let\oldsubparagraph\subparagraph
\renewcommand{\subparagraph}[1]{\oldsubparagraph{#1}\mbox{}}
\fi

%%% Use protect on footnotes to avoid problems with footnotes in titles
\let\rmarkdownfootnote\footnote%
\def\footnote{\protect\rmarkdownfootnote}

%%% Change title format to be more compact
\usepackage{titling}

% Create subtitle command for use in maketitle
\providecommand{\subtitle}[1]{
  \posttitle{
    \begin{center}\large#1\end{center}
    }
}

\setlength{\droptitle}{-2em}

  \title{Statlearn - homework II}
    \pretitle{\vspace{\droptitle}\centering\huge}
  \posttitle{\par}
    \author{}
    \preauthor{}\postauthor{}
    \date{}
    \predate{}\postdate{}
  

\begin{document}
\maketitle

\section{Part I - Song genre
classification}\label{part-i---song-genre-classification}

\subsection{Installing and importing
libraries}\label{installing-and-importing-libraries}

\begin{Shaded}
\begin{Highlighting}[]
\CommentTok{# this part is to be executed only once to install libraries we need }
\CommentTok{# i kindly suggest you run this on windows OS}
\CommentTok{# But if you feel like  solving R dependencies hell on linux... give it a try .}
\CommentTok{# about macOS , don't really know}
\CommentTok{# }
\CommentTok{# }
\CommentTok{# install.packages('signal')}
\CommentTok{# install.packages('audio')}
\CommentTok{# install.packages('wrassp')}
\CommentTok{# install.packages('warbleR')}
\CommentTok{# install.packages('tuneR')}
\CommentTok{# install.packages('audiolyzR')}
\end{Highlighting}
\end{Shaded}

\begin{Shaded}
\begin{Highlighting}[]
\CommentTok{# then we import all libraries needed here}
\KeywordTok{suppressMessages}\NormalTok{(}\KeywordTok{require}\NormalTok{(signal, }\DataTypeTok{quietly =}\NormalTok{ T))}
\KeywordTok{library}\NormalTok{(signal)}

\KeywordTok{suppressMessages}\NormalTok{(}\KeywordTok{require}\NormalTok{(audio, }\DataTypeTok{quietly =}\NormalTok{ T)) }
\KeywordTok{library}\NormalTok{(audio)}

\KeywordTok{suppressMessages}\NormalTok{(}\KeywordTok{require}\NormalTok{(wrassp,  }\DataTypeTok{quietly =}\NormalTok{ T))}
\KeywordTok{library}\NormalTok{(wrassp)}


\KeywordTok{library}\NormalTok{(warbleR)}
\end{Highlighting}
\end{Shaded}

\begin{verbatim}
## Loading required package: maps
\end{verbatim}

\begin{verbatim}
## Loading required package: tuneR
\end{verbatim}

\begin{verbatim}
## 
## Attaching package: 'tuneR'
\end{verbatim}

\begin{verbatim}
## The following object is masked from 'package:audio':
## 
##     play
\end{verbatim}

\begin{verbatim}
## Loading required package: seewave
\end{verbatim}

\begin{verbatim}
## 
## Attaching package: 'seewave'
\end{verbatim}

\begin{verbatim}
## The following object is masked from 'package:signal':
## 
##     unwrap
\end{verbatim}

\begin{verbatim}
## Loading required package: NatureSounds
\end{verbatim}

\begin{verbatim}
## 
## NOTE: functions are being renamed (run 'print(new_function_names)' to see new names). Both old and new names are available in this version 
##  Please see citation('warbleR') for use in publication
\end{verbatim}

\begin{Shaded}
\begin{Highlighting}[]
\KeywordTok{library}\NormalTok{(tuneR)}
\KeywordTok{library}\NormalTok{(audiolyzR)}
\end{Highlighting}
\end{Shaded}

\begin{verbatim}
## Loading required package: hexbin
\end{verbatim}

\begin{verbatim}
## Loading required package: RJSONIO
\end{verbatim}

\begin{verbatim}
## Loading required package: plotrix
\end{verbatim}

\begin{verbatim}
## 
## Attaching package: 'plotrix'
\end{verbatim}

\begin{verbatim}
## The following object is masked from 'package:seewave':
## 
##     rescale
\end{verbatim}

\subsection{Reading and describing
Data}\label{reading-and-describing-data}

\begin{Shaded}
\begin{Highlighting}[]
\CommentTok{# My path to the data }
\NormalTok{auPath <-}\StringTok{ "data_example"}

\CommentTok{# List the .au files}
\NormalTok{auFiles <-}\StringTok{ }\KeywordTok{list.files}\NormalTok{(auPath, }\DataTypeTok{pattern=}\KeywordTok{glob2rx}\NormalTok{(}\StringTok{'*.au'}\NormalTok{), }\DataTypeTok{full.names=}\OtherTok{TRUE}\NormalTok{)}
\NormalTok{auFiles}
\end{Highlighting}
\end{Shaded}

\begin{verbatim}
## [1] "data_example/f1.au" "data_example/f2.au"
\end{verbatim}

\begin{Shaded}
\begin{Highlighting}[]
\CommentTok{# Number of files }
\NormalTok{N <-}\StringTok{ }\KeywordTok{length}\NormalTok{(auFiles)}
\NormalTok{N}
\end{Highlighting}
\end{Shaded}

\begin{verbatim}
## [1] 2
\end{verbatim}

we have a total of \{N\} songs in our dataset .

\begin{Shaded}
\begin{Highlighting}[]
\KeywordTok{str}\NormalTok{(auFiles)}
\end{Highlighting}
\end{Shaded}

\begin{verbatim}
##  chr [1:2] "data_example/f1.au" "data_example/f2.au"
\end{verbatim}

\begin{Shaded}
\begin{Highlighting}[]
\CommentTok{#attributes(auFiles)}
\NormalTok{## Let's try to get file in order }
\NormalTok{ord =}\StringTok{ }\KeywordTok{c}\NormalTok{(}\DecValTok{1}\OperatorTok{:}\NormalTok{N)}
\NormalTok{ordFileList =}\StringTok{ }\KeywordTok{paste0}\NormalTok{(}\KeywordTok{rep}\NormalTok{(}\KeywordTok{paste0}\NormalTok{(auPath,}\StringTok{'/f'}\NormalTok{)),}\KeywordTok{paste0}\NormalTok{(ord,}\KeywordTok{rep}\NormalTok{(}\StringTok{'.au'}\NormalTok{,N)))}
\KeywordTok{str}\NormalTok{(ordFileList)}
\end{Highlighting}
\end{Shaded}

\begin{verbatim}
##  chr [1:2] "data_example/f1.au" "data_example/f2.au"
\end{verbatim}

\begin{Shaded}
\begin{Highlighting}[]
\CommentTok{# Load an audio file, e.g. the first one in the list above}
\NormalTok{x <-}\StringTok{ }\KeywordTok{read.AsspDataObj}\NormalTok{(ordFileList[}\DecValTok{1}\NormalTok{])}
\KeywordTok{str}\NormalTok{(}\KeywordTok{attributes}\NormalTok{(x))}
\end{Highlighting}
\end{Shaded}

\begin{verbatim}
## List of 10
##  $ names       : chr "audio"
##  $ trackFormats: chr "INT16"
##  $ sampleRate  : num 22050
##  $ filePath    : chr "data_example/f1.au"
##  $ origFreq    : num 0
##  $ startTime   : num 0
##  $ startRecord : int 1
##  $ endRecord   : int 666820
##  $ class       : chr "AsspDataObj"
##  $ fileInfo    : int [1:2] 15 2
\end{verbatim}

\begin{Shaded}
\begin{Highlighting}[]
\CommentTok{#Then we set a fixed samples length for all files}
\CommentTok{# as the minimum lenght of all of them }
\NormalTok{fixedLength =}\StringTok{ }\DecValTok{22050} \OperatorTok{*}\StringTok{ }\DecValTok{30} \CommentTok{# default length}


\ControlFlowTok{for}\NormalTok{ (i }\ControlFlowTok{in} \DecValTok{1}\OperatorTok{:}\NormalTok{N) \{}
\NormalTok{  x <-}\StringTok{ }\KeywordTok{read.AsspDataObj}\NormalTok{(auFiles[i])}
\NormalTok{  min =}\StringTok{ }\KeywordTok{attributes}\NormalTok{(x)}\OperatorTok{$}\NormalTok{endRecord   }\CommentTok{# the samples length of the current file}
\NormalTok{  fixedLength <-}\StringTok{ }\KeywordTok{ifelse}\NormalTok{(fixedLength}\OperatorTok{<=}\NormalTok{min, fixedLength, min) }\CommentTok{# we take the minimum}
\NormalTok{\}}

\NormalTok{fixedLength}
\end{Highlighting}
\end{Shaded}

\begin{verbatim}
## [1] 661500
\end{verbatim}

We can see from the output that The records were made at a sample rate
of 22050hz for a duration of 30 seconds and therefore contains around
22050 * 30 = 661500 samples

Now lets' plot rthe first file samples to geta general idea

\begin{Shaded}
\begin{Highlighting}[]
\CommentTok{# We }
\CommentTok{# (only plot every 10th element to accelerate plotting)}

\NormalTok{x =}\StringTok{ }\KeywordTok{read.AsspDataObj}\NormalTok{(ordFileList[}\DecValTok{1}\NormalTok{])}
\NormalTok{x}
\end{Highlighting}
\end{Shaded}

\begin{verbatim}
## Assp Data Object of file data_example/f1.au.
## Format: SND (binary)
## 666820 records at 22050 Hz
## Duration: 30.241270 s
## Number of tracks: 1 
##   audio (1 fields)
\end{verbatim}

\begin{Shaded}
\begin{Highlighting}[]
\NormalTok{ith =}\StringTok{ }\DecValTok{22050} \OperatorTok{/}\DecValTok{5} \CommentTok{# ith element to plot . basically we are plotting elements each 0.2s}

\NormalTok{x_axe =}\StringTok{ }\KeywordTok{seq}\NormalTok{(}\DecValTok{0}\NormalTok{,}\KeywordTok{numRecs.AsspDataObj}\NormalTok{(x) }\OperatorTok{-}\StringTok{ }\DecValTok{1}\NormalTok{, ith) }\OperatorTok{/}\StringTok{ }\KeywordTok{rate.AsspDataObj}\NormalTok{(x)}
\KeywordTok{length}\NormalTok{(x_axe)}
\end{Highlighting}
\end{Shaded}

\begin{verbatim}
## [1] 152
\end{verbatim}

\begin{Shaded}
\begin{Highlighting}[]
\NormalTok{y_axe =}\StringTok{  }\NormalTok{x}\OperatorTok{$}\NormalTok{audio[}\KeywordTok{c}\NormalTok{(}\OtherTok{TRUE}\NormalTok{, }\KeywordTok{rep}\NormalTok{(}\OtherTok{FALSE}\NormalTok{,ith}\OperatorTok{-}\DecValTok{1}\NormalTok{))]}
\KeywordTok{length}\NormalTok{(y_axe)}
\end{Highlighting}
\end{Shaded}

\begin{verbatim}
## [1] 152
\end{verbatim}

\begin{Shaded}
\begin{Highlighting}[]
\KeywordTok{plot}\NormalTok{(x_axe,}
\NormalTok{     y_axe,}
     \DataTypeTok{type=}\StringTok{'l'}\NormalTok{,}
     \DataTypeTok{xlab=}\StringTok{'time (s)'}\NormalTok{,}
     \DataTypeTok{ylab=}\StringTok{'Audio samples'}\NormalTok{)}
\end{Highlighting}
\end{Shaded}

\includegraphics{song_classification_g_20_files/figure-latex/unnamed-chunk-6-1.pdf}

\begin{Shaded}
\begin{Highlighting}[]
\KeywordTok{suppressMessages}\NormalTok{(}\KeywordTok{require}\NormalTok{(tuneR, }\DataTypeTok{quietly =}\NormalTok{ T))}
\NormalTok{?Wave}
\end{Highlighting}
\end{Shaded}

\begin{verbatim}
## starting httpd help server ... done
\end{verbatim}

\begin{Shaded}
\begin{Highlighting}[]
\NormalTok{?WaveMC}
\NormalTok{?getWavPlayer}
\NormalTok{?play}

\CommentTok{# Transform}
\NormalTok{xwv <-}\StringTok{ }\KeywordTok{Wave}\NormalTok{( }\KeywordTok{as.numeric}\NormalTok{(x}\OperatorTok{$}\NormalTok{audio[}\DecValTok{1}\OperatorTok{:}\NormalTok{fixedLength]), }\DataTypeTok{samp.rate =} \KeywordTok{rate.AsspDataObj}\NormalTok{(x), }\DataTypeTok{bit =} \DecValTok{16}\NormalTok{)}
\KeywordTok{setWavPlayer}\NormalTok{(}\StringTok{"afplay"}\NormalTok{) }\CommentTok{# needed for MAC OS not for Windoze}
\CommentTok{# play(xwv) # do not run in Markdown}
\KeywordTok{plot}\NormalTok{(xwv)}
\end{Highlighting}
\end{Shaded}

\includegraphics{song_classification_g_20_files/figure-latex/unnamed-chunk-7-1.pdf}

\begin{Shaded}
\begin{Highlighting}[]
\NormalTok{xw.dwn =}\StringTok{ }\KeywordTok{downsample}\NormalTok{(xwv, }\DataTypeTok{samp.rate =} \DecValTok{11025}\NormalTok{)}
\NormalTok{xwv}
\end{Highlighting}
\end{Shaded}

\begin{verbatim}
## 
## Wave Object
##  Number of Samples:      661500
##  Duration (seconds):     30
##  Samplingrate (Hertz):   22050
##  Channels (Mono/Stereo): Mono
##  PCM (integer format):   TRUE
##  Bit (8/16/24/32/64):    16
\end{verbatim}

\begin{Shaded}
\begin{Highlighting}[]
\NormalTok{xw.dwn}\OperatorTok{@}\NormalTok{samp.rate}
\end{Highlighting}
\end{Shaded}

\begin{verbatim}
## [1] 11025
\end{verbatim}

\begin{Shaded}
\begin{Highlighting}[]
\NormalTok{xw.dwn}
\end{Highlighting}
\end{Shaded}

\begin{verbatim}
## 
## Wave Object
##  Number of Samples:      330750
##  Duration (seconds):     30
##  Samplingrate (Hertz):   11025
##  Channels (Mono/Stereo): Mono
##  PCM (integer format):   TRUE
##  Bit (8/16/24/32/64):    16
\end{verbatim}

\begin{Shaded}
\begin{Highlighting}[]
\NormalTok{?attributes}

\KeywordTok{str}\NormalTok{(xwv)}
\end{Highlighting}
\end{Shaded}

\begin{verbatim}
## Formal class 'Wave' [package "tuneR"] with 6 slots
##   ..@ left     : num [1:661500] -13260 -10389 -8663 -8034 -8125 ...
##   ..@ right    : num(0) 
##   ..@ stereo   : logi FALSE
##   ..@ samp.rate: num 22050
##   ..@ bit      : num 16
##   ..@ pcm      : logi TRUE
\end{verbatim}

\begin{Shaded}
\begin{Highlighting}[]
\NormalTok{?powspec}
\NormalTok{out =}\StringTok{ }\KeywordTok{powspec}\NormalTok{( }\KeywordTok{as.numeric}\NormalTok{(x}\OperatorTok{$}\NormalTok{audio[}\DecValTok{1}\OperatorTok{:}\NormalTok{fixedLength]), }\DataTypeTok{sr =} \KeywordTok{rate.AsspDataObj}\NormalTok{(x))}
\CommentTok{# The output is a matrix, where each column represents a power spectrum }
\CommentTok{# for a given time frame and each row represents a frequency.}
\KeywordTok{dim}\NormalTok{(out)}
\end{Highlighting}
\end{Shaded}

\begin{verbatim}
## [1]  512 3005
\end{verbatim}

\begin{Shaded}
\begin{Highlighting}[]
\CommentTok{# }
\KeywordTok{image}\NormalTok{(out)}
\end{Highlighting}
\end{Shaded}

\includegraphics{song_classification_g_20_files/figure-latex/unnamed-chunk-8-1.pdf}

\begin{Shaded}
\begin{Highlighting}[]
\CommentTok{# Spetral info ------------------------------------------------------------}

\CommentTok{# calculate the fundamental frequency contour}
\NormalTok{f0vals =}\StringTok{ }\KeywordTok{ksvF0}\NormalTok{(}\StringTok{"data_example/f1.au"}\NormalTok{, }\DataTypeTok{toFile=}\NormalTok{F)}
\CommentTok{# plot the fundamental frequency contour}
\KeywordTok{plot}\NormalTok{(}\KeywordTok{seq}\NormalTok{(}\DecValTok{0}\NormalTok{,}\KeywordTok{numRecs.AsspDataObj}\NormalTok{(f0vals) }\OperatorTok{-}\StringTok{ }\DecValTok{1}\NormalTok{) }\OperatorTok{/}\StringTok{ }\KeywordTok{rate.AsspDataObj}\NormalTok{(f0vals) }\OperatorTok{+}
\StringTok{       }\KeywordTok{attr}\NormalTok{(f0vals, }\StringTok{'startTime'}\NormalTok{),}
\NormalTok{     f0vals}\OperatorTok{$}\NormalTok{F0, }
     \DataTypeTok{type=}\StringTok{'l'}\NormalTok{, }
     \DataTypeTok{xlab=}\StringTok{'time (s)'}\NormalTok{, }
     \DataTypeTok{ylab=}\StringTok{'F0 frequency (Hz)'}\NormalTok{)}
\end{Highlighting}
\end{Shaded}

\includegraphics{song_classification_g_20_files/figure-latex/unnamed-chunk-9-1.pdf}

\begin{Shaded}
\begin{Highlighting}[]
\CommentTok{# STFT --------------------------------------------------------------------}



\CommentTok{# Short Time Fourier Transform (default values)}
\CommentTok{# fs       <- rate.AsspDataObj(x)   # sampling rate}
\NormalTok{fs       <-}\StringTok{ }\NormalTok{xw.dwn}\OperatorTok{@}\NormalTok{samp.rate}
\NormalTok{winsize  <-}\StringTok{ }\DecValTok{2048}
\NormalTok{nfft     <-}\StringTok{ }\DecValTok{2048}
\NormalTok{hopsize  <-}\StringTok{ }\DecValTok{512}
\NormalTok{noverlap <-}\StringTok{ }\NormalTok{winsize }\OperatorTok{-}\StringTok{ }\NormalTok{hopsize}
\CommentTok{# sp  <- specgram(x = x$audio, n = nfft, Fs = fs, window = winsize, overlap = noverlap)}
\NormalTok{sp  <-}\StringTok{ }\KeywordTok{specgram}\NormalTok{(}\DataTypeTok{x =}\NormalTok{ xw.dwn}\OperatorTok{@}\NormalTok{left, }\DataTypeTok{n =}\NormalTok{ nfft, }\DataTypeTok{Fs =}\NormalTok{ fs, }\DataTypeTok{window =}\NormalTok{ winsize, }\DataTypeTok{overlap =}\NormalTok{ noverlap)}
\NormalTok{sp}
\end{Highlighting}
\end{Shaded}

\includegraphics{song_classification_g_20_files/figure-latex/unnamed-chunk-10-1.pdf}

\begin{Shaded}
\begin{Highlighting}[]
\KeywordTok{names}\NormalTok{(sp)}
\end{Highlighting}
\end{Shaded}

\begin{verbatim}
## [1] "S" "f" "t"
\end{verbatim}

\begin{Shaded}
\begin{Highlighting}[]
\KeywordTok{class}\NormalTok{(sp)}
\end{Highlighting}
\end{Shaded}

\begin{verbatim}
## [1] "specgram"
\end{verbatim}

\begin{Shaded}
\begin{Highlighting}[]
\CommentTok{# Setup -------------------------------------------------------------------}

\CommentTok{# STFT}
\NormalTok{winsize  <-}\StringTok{ }\DecValTok{2048}
\NormalTok{nfft     <-}\StringTok{ }\DecValTok{2048}
\NormalTok{hopsize  <-}\StringTok{ }\DecValTok{512}
\NormalTok{noverlap <-}\StringTok{ }\NormalTok{winsize }\OperatorTok{-}\StringTok{ }\NormalTok{hopsize}

\CommentTok{# Frequency bands selection}
\NormalTok{nb   <-}\StringTok{ }\DecValTok{2}\OperatorTok{^}\DecValTok{3}
\NormalTok{lowB <-}\StringTok{ }\DecValTok{100}
\NormalTok{eps  <-}\StringTok{ }\NormalTok{.Machine}\OperatorTok{$}\NormalTok{double.eps}

\CommentTok{# Number of seconds of the analyzed window}
\NormalTok{corrtime     <-}\StringTok{ }\DecValTok{15}
\end{Highlighting}
\end{Shaded}

\begin{Shaded}
\begin{Highlighting}[]
\CommentTok{# Analysis ----------------------------------------------------------------}

\CommentTok{# Sampling rate}
\CommentTok{# fs <- rate.AsspDataObj(x)}
\CommentTok{# Short-time fourier transform}
\CommentTok{# sp       <- specgram(x = x$audio, n = nfft, Fs = fs, window = winsize, overlap = noverlap)}
\NormalTok{ntm      <-}\StringTok{ }\KeywordTok{ncol}\NormalTok{(sp}\OperatorTok{$}\NormalTok{S)  }\CommentTok{# number of (overlapping) time segments}

\CommentTok{# Energy of bands}
\NormalTok{fco    <-}\StringTok{ }\KeywordTok{round}\NormalTok{( }\KeywordTok{c}\NormalTok{(}\DecValTok{0}\NormalTok{, lowB}\OperatorTok{*}\NormalTok{(fs}\OperatorTok{/}\DecValTok{2}\OperatorTok{/}\NormalTok{lowB)}\OperatorTok{^}\NormalTok{((}\DecValTok{0}\OperatorTok{:}\NormalTok{(nb}\OperatorTok{-}\DecValTok{1}\NormalTok{))}\OperatorTok{/}\NormalTok{(nb}\OperatorTok{-}\DecValTok{1}\NormalTok{)))}\OperatorTok{/}\NormalTok{fs}\OperatorTok{*}\NormalTok{nfft )}
\NormalTok{energy <-}\StringTok{ }\KeywordTok{matrix}\NormalTok{(}\DecValTok{0}\NormalTok{, nb, ntm)}
\ControlFlowTok{for}\NormalTok{ (tm }\ControlFlowTok{in} \DecValTok{1}\OperatorTok{:}\NormalTok{ntm)\{}
  \ControlFlowTok{for}\NormalTok{ (i }\ControlFlowTok{in} \DecValTok{1}\OperatorTok{:}\NormalTok{nb)\{}
\NormalTok{    lower_bound <-}\StringTok{ }\DecValTok{1} \OperatorTok{+}\StringTok{ }\NormalTok{fco[i]}
\NormalTok{    upper_bound <-}\StringTok{ }\KeywordTok{min}\NormalTok{( }\KeywordTok{c}\NormalTok{( }\DecValTok{1} \OperatorTok{+}\StringTok{ }\NormalTok{fco[i }\OperatorTok{+}\StringTok{ }\DecValTok{1}\NormalTok{], }\KeywordTok{nrow}\NormalTok{(sp}\OperatorTok{$}\NormalTok{S) ) )}
\NormalTok{    energy[i, tm] <-}\StringTok{ }\KeywordTok{sum}\NormalTok{( }\KeywordTok{abs}\NormalTok{(sp}\OperatorTok{$}\NormalTok{S[ lower_bound}\OperatorTok{:}\NormalTok{upper_bound, tm ])}\OperatorTok{^}\DecValTok{2}\NormalTok{ )}
\NormalTok{  \}}
\NormalTok{\}}
\NormalTok{energy[energy }\OperatorTok{<}\StringTok{ }\NormalTok{eps] <-}\StringTok{ }\NormalTok{eps}
\NormalTok{energy =}\StringTok{ }\DecValTok{10}\OperatorTok{*}\KeywordTok{log10}\NormalTok{(energy)}

\KeywordTok{dim}\NormalTok{(energy)}
\end{Highlighting}
\end{Shaded}

\begin{verbatim}
## [1]   8 642
\end{verbatim}

\begin{Shaded}
\begin{Highlighting}[]
\CommentTok{# Init pdf-plot}
\CommentTok{# pdf(file = paste(gsub(".wav","",nomefl),"-plot.pdf", sep = ""), height = 11, width = 8)}
\KeywordTok{par}\NormalTok{(}\DataTypeTok{mfrow =} \KeywordTok{c}\NormalTok{(}\DecValTok{2}\NormalTok{,}\DecValTok{1}\NormalTok{))}

\CommentTok{# Take a look}
\KeywordTok{matplot}\NormalTok{(sp}\OperatorTok{$}\NormalTok{t, }\KeywordTok{t}\NormalTok{(energy), }\DataTypeTok{type =} \StringTok{"l"}\NormalTok{, }\DataTypeTok{lty =} \DecValTok{1}\NormalTok{, }
        \DataTypeTok{main =} \StringTok{"Energy Band Envelopes"}\NormalTok{,}
        \DataTypeTok{xlab =} \StringTok{"Time (secs)"}\NormalTok{, }\DataTypeTok{ylab =} \StringTok{"Energy"}\NormalTok{, }\DataTypeTok{xaxt =} \StringTok{"n"}\NormalTok{,}
        \DataTypeTok{col =}\NormalTok{ viridis}\OperatorTok{::}\KeywordTok{viridis}\NormalTok{(nb}\OperatorTok{+}\DecValTok{1}\NormalTok{, .}\DecValTok{3}\NormalTok{), }\DataTypeTok{lwd =} \FloatTok{1.5}\NormalTok{)}
\KeywordTok{axis}\NormalTok{(}\DecValTok{1}\NormalTok{, }\KeywordTok{seq}\NormalTok{(}\KeywordTok{min}\NormalTok{(sp}\OperatorTok{$}\NormalTok{t), }\KeywordTok{max}\NormalTok{(sp}\OperatorTok{$}\NormalTok{t), }\DataTypeTok{length.out =} \DecValTok{15}\NormalTok{),}
     \KeywordTok{round}\NormalTok{(}\KeywordTok{seq}\NormalTok{(}\DecValTok{0}\NormalTok{,}\DecValTok{30}\NormalTok{,}\DataTypeTok{length.out =} \DecValTok{15}\NormalTok{)))}
\KeywordTok{legend}\NormalTok{( }\StringTok{"bottom"}\NormalTok{, }\KeywordTok{paste}\NormalTok{(}\StringTok{"Band-"}\NormalTok{,}\DecValTok{1}\OperatorTok{:}\NormalTok{nb,}\DataTypeTok{sep=}\StringTok{""}\NormalTok{), }\DataTypeTok{lwd =} \DecValTok{8}\NormalTok{, }\DataTypeTok{col =}\NormalTok{  viridis}\OperatorTok{::}\KeywordTok{viridis}\NormalTok{(nb}\OperatorTok{+}\DecValTok{1}\NormalTok{, .}\DecValTok{3}\NormalTok{),}
        \DataTypeTok{horiz =} \OtherTok{TRUE}\NormalTok{, }\DataTypeTok{bty =} \StringTok{"n"}\NormalTok{, }\DataTypeTok{cex =}\NormalTok{ .}\DecValTok{7}\NormalTok{ )}

\CommentTok{# 6sec zoom}
\KeywordTok{matplot}\NormalTok{(sp}\OperatorTok{$}\NormalTok{t, }\KeywordTok{t}\NormalTok{(energy), }\DataTypeTok{type =} \StringTok{"l"}\NormalTok{, }\DataTypeTok{lty =} \DecValTok{1}\NormalTok{, }
        \DataTypeTok{main =} \StringTok{"Energy Band Envelopes / 6s-Zoom"}\NormalTok{,}
        \DataTypeTok{xlab =} \StringTok{"Time (secs)"}\NormalTok{, }\DataTypeTok{ylab =} \StringTok{"Energy"}\NormalTok{, }\DataTypeTok{xaxt =} \StringTok{"n"}\NormalTok{, }\DataTypeTok{xlim =} \KeywordTok{c}\NormalTok{(}\DecValTok{0}\NormalTok{, }\FloatTok{6.5}\NormalTok{),}
        \DataTypeTok{col =}\NormalTok{ viridis}\OperatorTok{::}\KeywordTok{viridis}\NormalTok{(nb}\OperatorTok{+}\DecValTok{1}\NormalTok{, .}\DecValTok{3}\NormalTok{), }\DataTypeTok{lwd =} \FloatTok{1.5}\NormalTok{)}
\KeywordTok{axis}\NormalTok{(}\DecValTok{1}\NormalTok{, }\KeywordTok{seq}\NormalTok{(}\KeywordTok{min}\NormalTok{(sp}\OperatorTok{$}\NormalTok{t), }\KeywordTok{max}\NormalTok{(sp}\OperatorTok{$}\NormalTok{t), }\DataTypeTok{length.out =} \DecValTok{15}\NormalTok{), }
     \KeywordTok{round}\NormalTok{(}\KeywordTok{seq}\NormalTok{(}\DecValTok{0}\NormalTok{,}\DecValTok{30}\NormalTok{,}\DataTypeTok{length.out =} \DecValTok{15}\NormalTok{)))}
\KeywordTok{legend}\NormalTok{( }\StringTok{"bottom"}\NormalTok{, }\KeywordTok{paste}\NormalTok{(}\StringTok{"Band-"}\NormalTok{,}\DecValTok{1}\OperatorTok{:}\NormalTok{nb,}\DataTypeTok{sep=}\StringTok{""}\NormalTok{), }\DataTypeTok{lwd =} \DecValTok{8}\NormalTok{, }\DataTypeTok{col =}\NormalTok{  viridis}\OperatorTok{::}\KeywordTok{viridis}\NormalTok{(nb}\OperatorTok{+}\DecValTok{1}\NormalTok{, .}\DecValTok{3}\NormalTok{),}
        \DataTypeTok{horiz =} \OtherTok{TRUE}\NormalTok{, }\DataTypeTok{bty =} \StringTok{"n"}\NormalTok{, }\DataTypeTok{cex =}\NormalTok{ .}\DecValTok{7}\NormalTok{ )}
\end{Highlighting}
\end{Shaded}

\includegraphics{song_classification_g_20_files/figure-latex/unnamed-chunk-13-1.pdf}


\end{document}
