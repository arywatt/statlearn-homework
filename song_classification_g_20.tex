\documentclass[]{article}
\usepackage{lmodern}
\usepackage{amssymb,amsmath}
\usepackage{ifxetex,ifluatex}
\usepackage{fixltx2e} % provides \textsubscript
\ifnum 0\ifxetex 1\fi\ifluatex 1\fi=0 % if pdftex
  \usepackage[T1]{fontenc}
  \usepackage[utf8]{inputenc}
\else % if luatex or xelatex
  \ifxetex
    \usepackage{mathspec}
  \else
    \usepackage{fontspec}
  \fi
  \defaultfontfeatures{Ligatures=TeX,Scale=MatchLowercase}
\fi
% use upquote if available, for straight quotes in verbatim environments
\IfFileExists{upquote.sty}{\usepackage{upquote}}{}
% use microtype if available
\IfFileExists{microtype.sty}{%
\usepackage{microtype}
\UseMicrotypeSet[protrusion]{basicmath} % disable protrusion for tt fonts
}{}
\usepackage[margin=1in]{geometry}
\usepackage{hyperref}
\hypersetup{unicode=true,
            pdftitle={Statlearn - homework II},
            pdfborder={0 0 0},
            breaklinks=true}
\urlstyle{same}  % don't use monospace font for urls
\usepackage{color}
\usepackage{fancyvrb}
\newcommand{\VerbBar}{|}
\newcommand{\VERB}{\Verb[commandchars=\\\{\}]}
\DefineVerbatimEnvironment{Highlighting}{Verbatim}{commandchars=\\\{\}}
% Add ',fontsize=\small' for more characters per line
\usepackage{framed}
\definecolor{shadecolor}{RGB}{248,248,248}
\newenvironment{Shaded}{\begin{snugshade}}{\end{snugshade}}
\newcommand{\KeywordTok}[1]{\textcolor[rgb]{0.13,0.29,0.53}{\textbf{#1}}}
\newcommand{\DataTypeTok}[1]{\textcolor[rgb]{0.13,0.29,0.53}{#1}}
\newcommand{\DecValTok}[1]{\textcolor[rgb]{0.00,0.00,0.81}{#1}}
\newcommand{\BaseNTok}[1]{\textcolor[rgb]{0.00,0.00,0.81}{#1}}
\newcommand{\FloatTok}[1]{\textcolor[rgb]{0.00,0.00,0.81}{#1}}
\newcommand{\ConstantTok}[1]{\textcolor[rgb]{0.00,0.00,0.00}{#1}}
\newcommand{\CharTok}[1]{\textcolor[rgb]{0.31,0.60,0.02}{#1}}
\newcommand{\SpecialCharTok}[1]{\textcolor[rgb]{0.00,0.00,0.00}{#1}}
\newcommand{\StringTok}[1]{\textcolor[rgb]{0.31,0.60,0.02}{#1}}
\newcommand{\VerbatimStringTok}[1]{\textcolor[rgb]{0.31,0.60,0.02}{#1}}
\newcommand{\SpecialStringTok}[1]{\textcolor[rgb]{0.31,0.60,0.02}{#1}}
\newcommand{\ImportTok}[1]{#1}
\newcommand{\CommentTok}[1]{\textcolor[rgb]{0.56,0.35,0.01}{\textit{#1}}}
\newcommand{\DocumentationTok}[1]{\textcolor[rgb]{0.56,0.35,0.01}{\textbf{\textit{#1}}}}
\newcommand{\AnnotationTok}[1]{\textcolor[rgb]{0.56,0.35,0.01}{\textbf{\textit{#1}}}}
\newcommand{\CommentVarTok}[1]{\textcolor[rgb]{0.56,0.35,0.01}{\textbf{\textit{#1}}}}
\newcommand{\OtherTok}[1]{\textcolor[rgb]{0.56,0.35,0.01}{#1}}
\newcommand{\FunctionTok}[1]{\textcolor[rgb]{0.00,0.00,0.00}{#1}}
\newcommand{\VariableTok}[1]{\textcolor[rgb]{0.00,0.00,0.00}{#1}}
\newcommand{\ControlFlowTok}[1]{\textcolor[rgb]{0.13,0.29,0.53}{\textbf{#1}}}
\newcommand{\OperatorTok}[1]{\textcolor[rgb]{0.81,0.36,0.00}{\textbf{#1}}}
\newcommand{\BuiltInTok}[1]{#1}
\newcommand{\ExtensionTok}[1]{#1}
\newcommand{\PreprocessorTok}[1]{\textcolor[rgb]{0.56,0.35,0.01}{\textit{#1}}}
\newcommand{\AttributeTok}[1]{\textcolor[rgb]{0.77,0.63,0.00}{#1}}
\newcommand{\RegionMarkerTok}[1]{#1}
\newcommand{\InformationTok}[1]{\textcolor[rgb]{0.56,0.35,0.01}{\textbf{\textit{#1}}}}
\newcommand{\WarningTok}[1]{\textcolor[rgb]{0.56,0.35,0.01}{\textbf{\textit{#1}}}}
\newcommand{\AlertTok}[1]{\textcolor[rgb]{0.94,0.16,0.16}{#1}}
\newcommand{\ErrorTok}[1]{\textcolor[rgb]{0.64,0.00,0.00}{\textbf{#1}}}
\newcommand{\NormalTok}[1]{#1}
\usepackage{graphicx,grffile}
\makeatletter
\def\maxwidth{\ifdim\Gin@nat@width>\linewidth\linewidth\else\Gin@nat@width\fi}
\def\maxheight{\ifdim\Gin@nat@height>\textheight\textheight\else\Gin@nat@height\fi}
\makeatother
% Scale images if necessary, so that they will not overflow the page
% margins by default, and it is still possible to overwrite the defaults
% using explicit options in \includegraphics[width, height, ...]{}
\setkeys{Gin}{width=\maxwidth,height=\maxheight,keepaspectratio}
\IfFileExists{parskip.sty}{%
\usepackage{parskip}
}{% else
\setlength{\parindent}{0pt}
\setlength{\parskip}{6pt plus 2pt minus 1pt}
}
\setlength{\emergencystretch}{3em}  % prevent overfull lines
\providecommand{\tightlist}{%
  \setlength{\itemsep}{0pt}\setlength{\parskip}{0pt}}
\setcounter{secnumdepth}{0}
% Redefines (sub)paragraphs to behave more like sections
\ifx\paragraph\undefined\else
\let\oldparagraph\paragraph
\renewcommand{\paragraph}[1]{\oldparagraph{#1}\mbox{}}
\fi
\ifx\subparagraph\undefined\else
\let\oldsubparagraph\subparagraph
\renewcommand{\subparagraph}[1]{\oldsubparagraph{#1}\mbox{}}
\fi

%%% Use protect on footnotes to avoid problems with footnotes in titles
\let\rmarkdownfootnote\footnote%
\def\footnote{\protect\rmarkdownfootnote}

%%% Change title format to be more compact
\usepackage{titling}

% Create subtitle command for use in maketitle
\providecommand{\subtitle}[1]{
  \posttitle{
    \begin{center}\large#1\end{center}
    }
}

\setlength{\droptitle}{-2em}

  \title{Statlearn - homework II}
    \pretitle{\vspace{\droptitle}\centering\huge}
  \posttitle{\par}
    \author{}
    \preauthor{}\postauthor{}
    \date{}
    \predate{}\postdate{}
  

\begin{document}
\maketitle

\section{Part I - Song genre
classification}\label{part-i---song-genre-classification}

\subsection{Installing and importing
libraries}\label{installing-and-importing-libraries}

\begin{Shaded}
\begin{Highlighting}[]
\CommentTok{# this part is to be executed only once to install libraries we need }
\CommentTok{# i kindly suggest you run this on windows OS}
\CommentTok{# But if you feel like  solving R dependencies hell on linux... give it a try .}
\CommentTok{# about macOS , don't really know}
\CommentTok{# }
\CommentTok{# }
\CommentTok{# install.packages('signal')}
\CommentTok{# install.packages('audio')}
\CommentTok{# install.packages('wrassp')}
\CommentTok{# install.packages('warbleR')}
\CommentTok{# install.packages('tuneR')}
\CommentTok{# install.packages('audiolyzR')}
\end{Highlighting}
\end{Shaded}

\begin{Shaded}
\begin{Highlighting}[]
\CommentTok{# then we import all libraries needed here}
\KeywordTok{suppressMessages}\NormalTok{(}\KeywordTok{require}\NormalTok{(signal, }\DataTypeTok{quietly =}\NormalTok{ T))}
\KeywordTok{library}\NormalTok{(signal)}

\KeywordTok{suppressMessages}\NormalTok{(}\KeywordTok{require}\NormalTok{(audio, }\DataTypeTok{quietly =}\NormalTok{ T)) }
\KeywordTok{library}\NormalTok{(audio)}

\KeywordTok{suppressMessages}\NormalTok{(}\KeywordTok{require}\NormalTok{(wrassp,  }\DataTypeTok{quietly =}\NormalTok{ T))}
\KeywordTok{library}\NormalTok{(wrassp)}


\KeywordTok{library}\NormalTok{(warbleR)}
\end{Highlighting}
\end{Shaded}

\begin{verbatim}
## Loading required package: maps
\end{verbatim}

\begin{verbatim}
## Loading required package: tuneR
\end{verbatim}

\begin{verbatim}
## 
## Attaching package: 'tuneR'
\end{verbatim}

\begin{verbatim}
## The following object is masked from 'package:audio':
## 
##     play
\end{verbatim}

\begin{verbatim}
## Loading required package: seewave
\end{verbatim}

\begin{verbatim}
## 
## Attaching package: 'seewave'
\end{verbatim}

\begin{verbatim}
## The following object is masked from 'package:signal':
## 
##     unwrap
\end{verbatim}

\begin{verbatim}
## Loading required package: NatureSounds
\end{verbatim}

\begin{verbatim}
## 
## NOTE: functions are being renamed (run 'print(new_function_names)' to see new names). Both old and new names are available in this version 
##  Please see citation('warbleR') for use in publication
\end{verbatim}

\begin{Shaded}
\begin{Highlighting}[]
\KeywordTok{library}\NormalTok{(tuneR)}
\KeywordTok{library}\NormalTok{(audiolyzR)}
\end{Highlighting}
\end{Shaded}

\begin{verbatim}
## Loading required package: hexbin
\end{verbatim}

\begin{verbatim}
## Loading required package: RJSONIO
\end{verbatim}

\begin{verbatim}
## Loading required package: plotrix
\end{verbatim}

\begin{verbatim}
## 
## Attaching package: 'plotrix'
\end{verbatim}

\begin{verbatim}
## The following object is masked from 'package:seewave':
## 
##     rescale
\end{verbatim}

\subsection{Reading and describing
Data}\label{reading-and-describing-data}

\begin{Shaded}
\begin{Highlighting}[]
\CommentTok{# My path to the data }
\NormalTok{auPath <-}\StringTok{ "data_example"}
\NormalTok{labelsFile <-}\StringTok{ }\KeywordTok{paste0}\NormalTok{(auPath,}\StringTok{'/labels.txt'}\NormalTok{)}
\NormalTok{labelsFile}
\end{Highlighting}
\end{Shaded}

\begin{verbatim}
## [1] "data_example/labels.txt"
\end{verbatim}

\begin{Shaded}
\begin{Highlighting}[]
\CommentTok{# List the .au files}
\NormalTok{auFiles <-}\StringTok{ }\KeywordTok{list.files}\NormalTok{(auPath, }\DataTypeTok{pattern=}\KeywordTok{glob2rx}\NormalTok{(}\StringTok{'*.au'}\NormalTok{), }\DataTypeTok{full.names=}\OtherTok{TRUE}\NormalTok{)}
\NormalTok{auFiles}
\end{Highlighting}
\end{Shaded}

\begin{verbatim}
## [1] "data_example/f1.au" "data_example/f2.au"
\end{verbatim}

\begin{Shaded}
\begin{Highlighting}[]
\CommentTok{# Number of files }
\NormalTok{N <-}\StringTok{ }\KeywordTok{length}\NormalTok{(auFiles)}
\end{Highlighting}
\end{Shaded}

we have a total of \{N\} songs in our dataset .

\begin{Shaded}
\begin{Highlighting}[]
\NormalTok{## Let's try to get the files in order }
\NormalTok{ord =}\StringTok{ }\KeywordTok{c}\NormalTok{(}\DecValTok{1}\OperatorTok{:}\NormalTok{N)}
\NormalTok{ordFileList =}\StringTok{ }\KeywordTok{paste0}\NormalTok{(}\KeywordTok{rep}\NormalTok{(}\KeywordTok{paste0}\NormalTok{(auPath,}\StringTok{'/f'}\NormalTok{)),}\KeywordTok{paste0}\NormalTok{(ord,}\KeywordTok{rep}\NormalTok{(}\StringTok{'.au'}\NormalTok{,N)))}
\NormalTok{ordFileList}
\end{Highlighting}
\end{Shaded}

\begin{verbatim}
## [1] "data_example/f1.au" "data_example/f2.au"
\end{verbatim}

\begin{Shaded}
\begin{Highlighting}[]
\CommentTok{# let's get also labels for each files}
\NormalTok{labels <-}\StringTok{ }\KeywordTok{read.table}\NormalTok{(}\DataTypeTok{file=}\NormalTok{labelsFile, }\DataTypeTok{header=}\OtherTok{TRUE}\NormalTok{, }\DataTypeTok{sep=}\StringTok{" "}\NormalTok{,}\DataTypeTok{col.names  =} \KeywordTok{c}\NormalTok{(}\StringTok{'id'}\NormalTok{,}\StringTok{'type'}\NormalTok{))}
\end{Highlighting}
\end{Shaded}

\begin{verbatim}
## Warning in read.table(file = labelsFile, header = TRUE, sep = " ",
## col.names = c("id", : incomplete final line found by readTableHeader on
## 'data_example/labels.txt'
\end{verbatim}

\begin{verbatim}
## Warning in read.table(file = labelsFile, header = TRUE, sep = " ",
## col.names = c("id", : header and 'col.names' are of different lengths
\end{verbatim}

\begin{Shaded}
\begin{Highlighting}[]
\NormalTok{labels}
\end{Highlighting}
\end{Shaded}

\begin{verbatim}
##   id    type
## 1  1 country
## 2  2 country
\end{verbatim}

\begin{Shaded}
\begin{Highlighting}[]
\NormalTok{?read.csv}
\end{Highlighting}
\end{Shaded}

\begin{verbatim}
## starting httpd help server ... done
\end{verbatim}

\begin{Shaded}
\begin{Highlighting}[]
\CommentTok{# Load an audio file, e.g. the first one in the list above}
\NormalTok{x <-}\StringTok{ }\KeywordTok{read.AsspDataObj}\NormalTok{(ordFileList[}\DecValTok{1}\NormalTok{])}
\KeywordTok{str}\NormalTok{(}\KeywordTok{attributes}\NormalTok{(x))}
\end{Highlighting}
\end{Shaded}

\begin{verbatim}
## List of 10
##  $ names       : chr "audio"
##  $ trackFormats: chr "INT16"
##  $ sampleRate  : num 22050
##  $ filePath    : chr "data_example/f1.au"
##  $ origFreq    : num 0
##  $ startTime   : num 0
##  $ startRecord : int 1
##  $ endRecord   : int 666820
##  $ class       : chr "AsspDataObj"
##  $ fileInfo    : int [1:2] 15 2
\end{verbatim}

\begin{Shaded}
\begin{Highlighting}[]
\CommentTok{#Then we set a fixed samples length for all files}
\CommentTok{# as the minimum lenght of all of them }
\NormalTok{fixedLength =}\StringTok{ }\DecValTok{22050} \OperatorTok{*}\StringTok{ }\DecValTok{30} \CommentTok{# default length}


\ControlFlowTok{for}\NormalTok{ (i }\ControlFlowTok{in} \DecValTok{1}\OperatorTok{:}\NormalTok{N) \{}
\NormalTok{  x <-}\StringTok{ }\KeywordTok{read.AsspDataObj}\NormalTok{(auFiles[i])}
\NormalTok{  min =}\StringTok{ }\KeywordTok{attributes}\NormalTok{(x)}\OperatorTok{$}\NormalTok{endRecord   }\CommentTok{# the samples length of the current file}
\NormalTok{  fixedLength <-}\StringTok{ }\KeywordTok{ifelse}\NormalTok{(fixedLength}\OperatorTok{<=}\NormalTok{min, fixedLength, min) }\CommentTok{# we take the minimum}
\NormalTok{\}}
\end{Highlighting}
\end{Shaded}

We can see from the output that The records were made at a sample rate
of 22050hz for a duration of 30 seconds and therefore contains around
22050 * 30 = 661500 samples

Now lets' plot rthe first file samples to geta general idea

\begin{Shaded}
\begin{Highlighting}[]
\CommentTok{# We }
\CommentTok{# (only plot every 10th element to accelerate plotting)}

\NormalTok{x =}\StringTok{ }\KeywordTok{read.AsspDataObj}\NormalTok{(ordFileList[}\DecValTok{1}\NormalTok{])}
\NormalTok{x}
\end{Highlighting}
\end{Shaded}

\begin{verbatim}
## Assp Data Object of file data_example/f1.au.
## Format: SND (binary)
## 666820 records at 22050 Hz
## Duration: 30.241270 s
## Number of tracks: 1 
##   audio (1 fields)
\end{verbatim}

\begin{Shaded}
\begin{Highlighting}[]
\NormalTok{ith =}\StringTok{ }\DecValTok{22050} \OperatorTok{/}\DecValTok{5} \CommentTok{# ith element to plot . basically we are plotting elements each 0.2s}

\NormalTok{x_axe =}\StringTok{ }\KeywordTok{seq}\NormalTok{(}\DecValTok{0}\NormalTok{,}\KeywordTok{numRecs.AsspDataObj}\NormalTok{(x) }\OperatorTok{-}\StringTok{ }\DecValTok{1}\NormalTok{, ith) }\OperatorTok{/}\StringTok{ }\KeywordTok{rate.AsspDataObj}\NormalTok{(x)}

\NormalTok{y_axe =}\StringTok{  }\NormalTok{x}\OperatorTok{$}\NormalTok{audio[}\KeywordTok{c}\NormalTok{(}\OtherTok{TRUE}\NormalTok{, }\KeywordTok{rep}\NormalTok{(}\OtherTok{FALSE}\NormalTok{,ith}\OperatorTok{-}\DecValTok{1}\NormalTok{))]}

\KeywordTok{plot}\NormalTok{(x_axe,}
\NormalTok{     y_axe,}
     \DataTypeTok{type=}\StringTok{'l'}\NormalTok{,}
     \DataTypeTok{xlab=}\StringTok{'time (s)'}\NormalTok{,}
     \DataTypeTok{ylab=}\StringTok{'Audio samples'}\NormalTok{)}
\end{Highlighting}
\end{Shaded}

\includegraphics{song_classification_g_20_files/figure-latex/unnamed-chunk-6-1.pdf}

\subsection{Features Extractions}\label{features-extractions}

\subsubsection{Features to extract}\label{features-to-extract}

\paragraph{Zero crossing rate}\label{zero-crossing-rate}

The zero-crossing rate is the rate of sign-changes along a signal, i.e.,
the rate at which the signal changes from positive to zero to negative
or from negative to zero to positive.{[}1{]} This feature has been used
heavily in both speech recognition and music information retrieval,
being a key feature to classify percussive sounds.{[}2{]}

ZCR is defined formally as

\{\displaystyle zcr=\{\frac {1}{T-1}\}\sum \emph{\{t=1\}\^{}\{T-1\}\mathbb {1}
}\{\mathbb {R} \emph{\{\textless{}0\}\}(s}\{t\}s\_\{t-1\})\}
\{\displaystyle zcr=\{\frac {1}{T-1}\}\sum \emph{\{t=1\}\^{}\{T-1\}\mathbb {1}
}\{\mathbb {R} \emph{\{\textless{}0\}\}(s}\{t\}s\_\{t-1\})\} where
\{\displaystyle s\} s is a signal of length \{\displaystyle T\} T and
\{\displaystyle \mathbb {1} \emph{\{\mathbb {R} }\{\textless{}0\}\}\}
\{\displaystyle \mathbb {1} \emph{\{\mathbb {R} }\{\textless{}0\}\}\}

is an indicator function.

\paragraph{Spectral properties}\label{spectral-properties}

the spectral prperties are a set of statistics computed on the spectrum
of an audio signal, sucha as - Spectral Centroid ( most important one) -
Spectral mean or median - spectral quartiles

The spectral centroid is a measure used in digital signal processing to
characterise a spectrum. It indicates where the center of mass of the
spectrum is located. Perceptually, it has a robust connection with the
impression of brightness of a sound. We basically loop on on audio
files, and compute some features using functions defined above .

It is calculated as the weighted mean of the frequencies present in the
signal, determined using a Fourier transform, with their magnitudes as
the weights

\paragraph{Spectral roll Off}\label{spectral-roll-off}

The roll-off frequency is defined as the frequency under which some
percentage (cutoff) of the total energy of the spectrum is contained.
The roll-off frequency can be used to distinguish between harmonic
(below roll-off) and noisy sounds (above roll-off).

\paragraph{Mel Frequency Cepstral
Coefficients}\label{mel-frequency-cepstral-coefficients}

Mel Frequency Cepstral Coefficients (MFCC) for an object of class Wave.
In speech recognition MFCCs are used to extract the stimulus of the
vocal tract from speech

\paragraph{Chroma frequencies}\label{chroma-frequencies}

Chroma features are an interesting and powerful representation for music
audio in which the entire spectrum is projected onto 12 bins
representing the 12 distinct semitones (or chroma) of the musical
octave. Since, in music, notes exactly one octave apart are perceived as
particularly similar, knowing the distribution of chroma even without
the absolute frequency (i.e.~the original octave) can give useful
musical information about the audio -- and may even reveal perceived
musical similarity that is not apparent in the original spectra.

\subsubsection{Implementation with R}\label{implementation-with-r}

In R we used principally package - Seawave - SoundGen - TuneR

\paragraph{Convert audio data to wave}\label{convert-audio-data-to-wave}

\begin{Shaded}
\begin{Highlighting}[]
\CommentTok{# Transform}


\CommentTok{# x : array to transform }
\CommentTok{# rate : the sample rate of x }
\CommentTok{# bit : }
\CommentTok{# reduceRate : wether reduce the sample rate or not }
\CommentTok{# newRate  # if down Sample is TRUE, new sample rate to use }
\NormalTok{transformToWave <-}\StringTok{ }\ControlFlowTok{function}\NormalTok{(x, rate, }\DataTypeTok{bit =} \DecValTok{16}\NormalTok{,}\DataTypeTok{reduceRate =} \OtherTok{FALSE}\NormalTok{, }\DataTypeTok{newRate =} \DecValTok{11025}\NormalTok{ )\{}
\NormalTok{  xwv =}\StringTok{ }\KeywordTok{Wave}\NormalTok{( }\KeywordTok{as.numeric}\NormalTok{(x), }\DataTypeTok{samp.rate =}\NormalTok{ rate, }\DataTypeTok{bit =}\NormalTok{ bit)}
  \ControlFlowTok{if}\NormalTok{(reduceRate)\{}
\NormalTok{    xwv =}\StringTok{ }\KeywordTok{downsample}\NormalTok{(xwv, }\DataTypeTok{samp.rate =}\NormalTok{ newRate)}
\NormalTok{  \}}
  \CommentTok{#transformedWave <- ifelse(reduceRate,  downsample(xwv, samp.rate = newRate),xwv)}
  \KeywordTok{return}\NormalTok{( xwv)}
\NormalTok{\}}
\end{Highlighting}
\end{Shaded}

\paragraph{Spectrum analysis , power spectrum and energy
band}\label{spectrum-analysis-power-spectrum-and-energy-band}

\begin{Shaded}
\begin{Highlighting}[]
\CommentTok{# Compute the powerspectrum of the input signal}

\CommentTok{# x : audio samples array }
\CommentTok{# rate : samples rate }
\CommentTok{# The output is a matrix, where each column represents a power spectrum }
\CommentTok{# for a given time frame and each row represents a frequency.}
\NormalTok{powerSpectrum <-}\StringTok{ }\ControlFlowTok{function}\NormalTok{(x , rate )\{}
\NormalTok{  out =}\StringTok{ }\KeywordTok{powspec}\NormalTok{( }\KeywordTok{as.numeric}\NormalTok{(x), rate)}
  \KeywordTok{return}\NormalTok{(out)}

\NormalTok{\}}


\CommentTok{# Spectral info ------------------------------------------------------------}

\CommentTok{# calculate the fundamental frequency contour}
\CommentTok{# name : name of the file input}
\NormalTok{spectralInfo <-}\StringTok{ }\ControlFlowTok{function}\NormalTok{(name)\{}
\NormalTok{   f0vals =}\StringTok{ }\KeywordTok{ksvF0}\NormalTok{(}\StringTok{"data_example/f1.au"}\NormalTok{, }\DataTypeTok{toFile=}\NormalTok{F)}
   \KeywordTok{return}\NormalTok{(f0vals)}
\NormalTok{\}}



\CommentTok{# --------------------------------------------------------------------}

\CommentTok{# Get the spectogram of a wave }
\CommentTok{# x : wave array }
\CommentTok{# winsize : Fourier transform window size}
\CommentTok{# fs : rate }
\CommentTok{# overlap : overlap with previous window, defaults to half the window length.}
\NormalTok{getSpecgram <-}\StringTok{ }\ControlFlowTok{function}\NormalTok{ (x, winsize, fs ,overlap)\{}
\NormalTok{  sp  <-}\StringTok{ }\KeywordTok{specgram}\NormalTok{(x, }\DataTypeTok{n =}\NormalTok{ winsize, }\DataTypeTok{Fs =}\NormalTok{ fs,  }\DataTypeTok{overlap =}\NormalTok{ overlap)}
  \KeywordTok{return}\NormalTok{(sp)}
  
\NormalTok{\}}

\CommentTok{#---------------------------------------------------------------------}

\CommentTok{#Frequency spectrum of a time wave}

\CommentTok{# x : an R object.}

\CommentTok{# fs :  sampling frequency of wave (in Hz). Does not need to be specified if embedded in wave.}

\CommentTok{# wl    : if at is not null, length of the window for the analysis (by default = 512).}

\CommentTok{# wn     : window name, see ftwindow (by default "hanning").}

\CommentTok{# fftw : if TRUE calls the function FFT of the library fftw for faster computation. See Notes of the function spectro.}

\CommentTok{# norm  #if TRUE the spectrum is normalised by its maximum.}

\NormalTok{getSpec <-}\StringTok{ }\ControlFlowTok{function}\NormalTok{ (x, winsize, fs )\{}
\NormalTok{  sp  <-}\StringTok{ }\KeywordTok{spec}\NormalTok{(x,  }\DataTypeTok{fs =}\NormalTok{ fs, }\DataTypeTok{wl =}\NormalTok{ winsize, }\DataTypeTok{fftw =} \OtherTok{TRUE}\NormalTok{,}\DataTypeTok{norm  =}\OtherTok{TRUE}\NormalTok{)}
  \KeywordTok{return}\NormalTok{(sp)}
\NormalTok{\}}

\CommentTok{#-----------------------------------------------------------------------------------}
\CommentTok{# To get spectral properties }

\CommentTok{# spec  : a data set resulting of a spectral analysis obtained with spec or meanspec (not in dB).}

\CommentTok{# f  :sampling frequency of spec (in Hz).}

\CommentTok{# str   :logical, if TRUE returns the results in a structured table.}

\CommentTok{# flim  :a vector of length 2 to specifgy the frequency limits of the analysis (in kHz)}

\CommentTok{# mel   #a logical, if TRUE the (htk-)mel scale is used.}

\NormalTok{GetSpecProps <-}\StringTok{ }\ControlFlowTok{function}\NormalTok{(x, fs)\{}
\NormalTok{  specProps =}\StringTok{ }\KeywordTok{specprop}\NormalTok{(x, }\DataTypeTok{f=}\NormalTok{ fs, }\DataTypeTok{mel =} \OtherTok{TRUE}\NormalTok{)}
  \KeywordTok{return}\NormalTok{(specProps)}
\NormalTok{\}}

\CommentTok{#---------------------------------------------------------------}

\CommentTok{# compute the zero crossing rate}

\CommentTok{# x :  R wave object    }


\CommentTok{#f :sampling frequency of wave (in Hz). Does not need to be specified if embedded in wave.}

\CommentTok{#wl: length of the window for the analysis (even number of points, by default = 512). If NULL the zero-crossing rate is computed of the complete signal.}

\CommentTok{#overlap    : overlap between two successive analysis windows (in %) if wl is not NULL.}
\NormalTok{  zeroCrossingRate <-}\StringTok{ }\ControlFlowTok{function}\NormalTok{ (x , fs, wl , overlap )\{}
    
\NormalTok{    cr =}\StringTok{ }\KeywordTok{zcr}\NormalTok{(x,}\DataTypeTok{f=}\NormalTok{ fs, }\DataTypeTok{wl =}\NormalTok{ wl, }\DataTypeTok{ovlp =}\NormalTok{ overlap)}
    \KeywordTok{return}\NormalTok{(zcr)}
    
\NormalTok{  \}}

\CommentTok{# Computation of MFCCs (Mel Frequency Cepstral Coefficients) for a Wave object}

\CommentTok{# x :  Object of class Wave.}

\NormalTok{getMfccs <-}\StringTok{ }\ControlFlowTok{function}\NormalTok{(x)\{}
\NormalTok{  mfccs =}\StringTok{ }\KeywordTok{MFCC}\NormalTok{(x, }\DataTypeTok{a =} \FloatTok{0.1}\NormalTok{, }\DataTypeTok{HW.width =} \FloatTok{0.025}\NormalTok{, }\DataTypeTok{HW.overlapping =} \FloatTok{0.25}\NormalTok{, }
    \DataTypeTok{T.number =} \DecValTok{24}\NormalTok{, }\DataTypeTok{T.overlapping =} \FloatTok{0.5}\NormalTok{, }\DataTypeTok{K =} \DecValTok{12}\NormalTok{)}
  
  \KeywordTok{return}\NormalTok{(mfcccs)}
\NormalTok{\}}


\CommentTok{# Energy bands  ----------------------------------------------------------------}

\CommentTok{# x : audio spectogram}
\CommentTok{# winsize : Fourier transform window size}
\CommentTok{# fs : rate }
\CommentTok{# nb : number of bands to select}
\CommentTok{# lowB :}
\CommentTok{# eps : default minimum energy value }



\NormalTok{energyBands <-}\StringTok{ }\ControlFlowTok{function}\NormalTok{(x,fs,nb, lowB,eps,winsize)\{}
\NormalTok{ntm      <-}\StringTok{ }\KeywordTok{ncol}\NormalTok{(x}\OperatorTok{$}\NormalTok{S)  }\CommentTok{# number of (overlapping) time segments}
\NormalTok{fco    <-}\StringTok{ }\KeywordTok{round}\NormalTok{( }\KeywordTok{c}\NormalTok{(}\DecValTok{0}\NormalTok{, lowB}\OperatorTok{*}\NormalTok{(fs}\OperatorTok{/}\DecValTok{2}\OperatorTok{/}\NormalTok{lowB)}\OperatorTok{^}\NormalTok{((}\DecValTok{0}\OperatorTok{:}\NormalTok{(nb}\OperatorTok{-}\DecValTok{1}\NormalTok{))}\OperatorTok{/}\NormalTok{(nb}\OperatorTok{-}\DecValTok{1}\NormalTok{)))}\OperatorTok{/}\NormalTok{fs}\OperatorTok{*}\NormalTok{winsize )}
\NormalTok{energy <-}\StringTok{ }\KeywordTok{matrix}\NormalTok{(}\DecValTok{0}\NormalTok{, nb, ntm)}
\ControlFlowTok{for}\NormalTok{ (tm }\ControlFlowTok{in} \DecValTok{1}\OperatorTok{:}\NormalTok{ntm)\{}
  \ControlFlowTok{for}\NormalTok{ (i }\ControlFlowTok{in} \DecValTok{1}\OperatorTok{:}\NormalTok{nb)\{}
\NormalTok{    lower_bound <-}\StringTok{ }\DecValTok{1} \OperatorTok{+}\StringTok{ }\NormalTok{fco[i]}
\NormalTok{    upper_bound <-}\StringTok{ }\KeywordTok{min}\NormalTok{( }\KeywordTok{c}\NormalTok{( }\DecValTok{1} \OperatorTok{+}\StringTok{ }\NormalTok{fco[i }\OperatorTok{+}\StringTok{ }\DecValTok{1}\NormalTok{], }\KeywordTok{nrow}\NormalTok{(x}\OperatorTok{$}\NormalTok{S) ) )}
\NormalTok{    energy[i, tm] <-}\StringTok{ }\KeywordTok{sum}\NormalTok{( }\KeywordTok{abs}\NormalTok{(x}\OperatorTok{$}\NormalTok{S[ lower_bound}\OperatorTok{:}\NormalTok{upper_bound, tm ])}\OperatorTok{^}\DecValTok{2}\NormalTok{ )}
\NormalTok{  \}}
\NormalTok{\}}
\NormalTok{energy[energy }\OperatorTok{<}\StringTok{ }\NormalTok{eps] <-}\StringTok{ }\NormalTok{eps}
\NormalTok{energy =}\StringTok{ }\DecValTok{10}\OperatorTok{*}\KeywordTok{log10}\NormalTok{(energy)}

\KeywordTok{return}\NormalTok{(energy)}
  
  
\NormalTok{\}}
\end{Highlighting}
\end{Shaded}

\subsubsection{Dataset Creation}\label{dataset-creation}

Basically we loop over the audio file, extracting all features and
saving in in text file

\begin{Shaded}
\begin{Highlighting}[]
\CommentTok{# first we define the general parameters}

\NormalTok{rate =}\StringTok{ }\DecValTok{22050}
\NormalTok{newrate =}\StringTok{ }\DecValTok{11050}
\NormalTok{reduceRate =}\StringTok{ }\OtherTok{FALSE}


\CommentTok{# STFT}
\NormalTok{winsize  <-}\StringTok{ }\DecValTok{2048}
\NormalTok{nfft     <-}\StringTok{ }\DecValTok{2048}
\NormalTok{hopsize  <-}\StringTok{ }\DecValTok{512}
\NormalTok{overlap <-}\StringTok{ }\NormalTok{winsize }\OperatorTok{-}\StringTok{ }\NormalTok{hopsize}

\CommentTok{# Frequency bands selection}
\NormalTok{nb   <-}\StringTok{ }\DecValTok{2}\OperatorTok{^}\DecValTok{3}
\NormalTok{lowB <-}\StringTok{ }\DecValTok{100}
\NormalTok{eps  <-}\StringTok{ }\NormalTok{.Machine}\OperatorTok{$}\NormalTok{double.eps}
\CommentTok{# Number of seconds of the analyzed window}
\NormalTok{corrtime     <-}\StringTok{ }\DecValTok{15}

\CommentTok{# the file list to use is our Ordered file list }

\NormalTok{data =}\StringTok{ }\KeywordTok{list}\NormalTok{()}

\ControlFlowTok{for}\NormalTok{( file }\ControlFlowTok{in}\NormalTok{ ordFileList)\{}
  
  \CommentTok{# we only take the fixed length sample  to make equal for all files }
\NormalTok{  x =}\StringTok{ }\KeywordTok{read.AsspDataObj}\NormalTok{(file)}\OperatorTok{$}\NormalTok{audio[}\DecValTok{1}\OperatorTok{:}\NormalTok{fixedLength]}
\NormalTok{  xWave =}\StringTok{ }\KeywordTok{transformToWave}\NormalTok{(x,rate)}
\NormalTok{  xPoweSpec =}\StringTok{ }\KeywordTok{powerSpectrum}\NormalTok{(x,rate)}
\NormalTok{  xSpecgram =}\StringTok{ }\KeywordTok{getSpecgram}\NormalTok{(x,}\DataTypeTok{fs =}\NormalTok{ rate,}\DataTypeTok{winsize =}\NormalTok{ winsize,}\DataTypeTok{overlap =}\NormalTok{ overlap)}
\NormalTok{  xSpectralInfo =}\StringTok{ }\KeywordTok{spectralInfo}\NormalTok{(file)}
  
  
  \CommentTok{# we save all the values }
  
  \CommentTok{#xData = data.frame(xWave,xPoweSpec,xSpecgram,xSpectralInfo)}
  \CommentTok{#data <- c(data,xData)}
  
\NormalTok{\}}
\end{Highlighting}
\end{Shaded}

\subsection{Models to apply}\label{models-to-apply}

\subsubsection{List of models to use and reasons (limit to models seen
in
class)}\label{list-of-models-to-use-and-reasons-limit-to-models-seen-in-class}

\subsubsection{Models Implementation}\label{models-implementation}

\subsection{classification}\label{classification}

\subsubsection{Classification performance for each
model}\label{classification-performance-for-each-model}

\subsubsection{Chose the best model}\label{chose-the-best-model}

\subsubsection{Determine each feature contribution to
model}\label{determine-each-feature-contribution-to-model}

\subsubsection{Maintain only the most important
ones}\label{maintain-only-the-most-important-ones}

\subsubsection{Final classification}\label{final-classification}

\subsection{Map}\label{map}

\section{Part II - Theory}\label{part-ii---theory}


\end{document}
